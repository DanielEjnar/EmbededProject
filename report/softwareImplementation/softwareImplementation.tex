\section{User application in C++}

This section will briefly explain how we implemented the user application in C++ on FreeRTOS. 

The program is used from the terminal where the user creates actions using standard input. Two concurrent OSAPI threads are created, one to handle user input and one to create the random numbers continually. Figure \ref{lst:mainapp} is a short code snippet of the main application.

\begin{lstlisting}[style=customc++,caption={Main application, where two threads are created and the scheduler started.},label={lst:mainapp}]
int main()
{
	//Create threads
	UserThread mUserThread(Thread::PRIORITY_ABOVE_NORMAL, "UserThread");
	RandomThread mRandomhread(Thread::PRIORITY_NORMAL, "RandomThread");

	/* Start FreeRTOS, the tasks running. */
	vTaskStartScheduler();
	for( ;; );
	return 0;
}
\end{lstlisting}

The $UserThread$ is the main point of interest, as it has a virtual function $run()$ executed by FreeRTOS and shown in listing \ref{lst:mainrun}. We chose to omit the $RandomThread$ from this review, however the code can be found in the attached code (userapp/RandomThread.h). We create a smart pointer to the \textbf{Context} class, as the client interacts solely with the states and data through this. Using a smart pointer improves safety of the application in case of exceptions and it gets deleted automatically when exiting the scope. 

Given a certain input, we create a pointer of class \textbf{Action} with a sting of what is to be done. This is evaluated by the current state of the application, as seen in the example in listing \ref{lst:mainrun}.

\begin{lstlisting}[style=customc++,caption={Main application, where two threads are created with a scheduler.},label={lst:mainrun}]
{
	std::unique_ptr<Context> context = std::make_unique<Context>();
	bool running = true;
	char* str;
	while(running) {
		std::cout << "Welcome to the menu" << std::endl;
		std::cout << "Press 1 for SETUP, 2 for RUN, 3 for ABORT" << std::endl;
		std::cin >> str;
		int num = atoi(str);
		switch(num) {
			case 1:
				context->HandleInput(std::make_unique<Action>("ENTER_SETUP"));
				context->HandleInput(std::make_unique<Action>("SETUP_DONE"));
				break;
			case 2:
				context->HandleInput(std::make_unique<Action>("OPTIMIZE"));
				break;
			case 3:
				context->HandleInput(std::make_unique<Action>("ABORT"));
		}
	}
}
\end{lstlisting}

The \textbf{Context} contains various behaviors, but of most interest is the $HandleInput()$ function. It takes an action from user space, delegates it to the current state and expects a pointer returned to an eventual new state. The current state is then updated with the returned state, in between calls to $Exit()$ and $Enter()$. Listing \ref{lst:context} gives an example in C++ code.

\begin{lstlisting}[style=customc++,caption={The Context class holds the current state among other variables, such as best chromosome so far and parameters. Here is showed the function HandleInput() called from user side.},label={lst:context}]
void Context::HandleInput(std::unique_ptr<Action> action) {
	std::unique_ptr<State> newState = _currentState->HandleAction(*this,
		std::move(action));
	if (newState != NULL) {
		_currentState->Exit(*this);
		_currentState.reset();
		_currentState = std::move(newState);
		_currentState->Enter(*this); }
}
\end{lstlisting}

To complete our example, we review the $HandleAction()$ function in the \textbf{Idle} class (listing \ref{lst:stateidle}). It takes the Context and action as parameters, checks the content of the action and acts accordingly by returning a pointer to the next state the application will transition to, in this case \textbf{Setup}.

\begin{lstlisting}[style=customc++,caption={The HandleAction() function in state Idle.},label={lst:stateidle}]
std::unique_ptr<State> Idle::HandleAction(Context& context, std::unique_ptr<Action> action)
{
	if((*action).GetAction() == "ENTER_SETUP") {
		std::cout << "EnterSetup() called" << std::endl;
		action.reset();
		return std::make_unique<Setup>();
	}
	...
	return NULL;
}
\end{lstlisting}

