\section{Code}

In the following the code for the testbenches and the stimulus are shown.

\begin{lstlisting}[style=customc++,caption={Testbench for GenerationGenerator},label={lst:generatorTestbench}]
#ifdef _MSC_VER
	#include "stdafx.h"
#endif
#include <systemc.h>
#include "GenerationGenerator.h"
#include "Stim.h"

#define CLK_PERIODE   20//ns
int sc_main(int argc, char* argv[]) {
  GenerationGenerator GenerationGenerator("GenerationGenerator");
  Stim Stim("Stim");
  
  sc_clock clock("clock", sc_time(CLK_PERIODE, SC_NS)); // 50 MHz
  sc_signal<bool> reset("reset");

  sc_signal<sc_uint<CHROMOSOME_WIDTH> > generation_parent1_in_channel;
  sc_signal<sc_uint<CHROMOSOME_WIDTH> > generation_parent2_in_channel;
  sc_signal<sc_uint<CHROMOSOME_WIDTH> > generation_child1_out_channel;
  sc_signal<sc_uint<CHROMOSOME_WIDTH> > generation_child2_out_channel;
  sc_signal<sc_uint<RANDOM_WIDTH> > mutation_probability_in_channel("mutation_probability_in_channel");
  sc_signal<sc_uint<RANDOM_WIDTH> > random_channel("random_channel");
  sc_signal<bool> generatingDone;
  sc_signal<bool> startGenerating;

  // Wire GenerationGeneraton
  GenerationGenerator.generation_parent1(generation_parent1_in_channel);
  GenerationGenerator.generation_parent2(generation_parent2_in_channel);
  GenerationGenerator.generation_child1(generation_child1_out_channel);
  GenerationGenerator.generation_child2(generation_child2_out_channel);
  GenerationGenerator.clk(clock);
  GenerationGenerator.mutation_probability(mutation_probability_in_channel);
  GenerationGenerator.random(random_channel);
  GenerationGenerator.reset(reset);
  GenerationGenerator.startGenerating(startGenerating);
  GenerationGenerator.generatingDone(generatingDone);

  // Wire Stim
  Stim.generation_parent1(generation_parent1_in_channel);
  Stim.generation_parent2(generation_parent2_in_channel);
  Stim.clk(clock);
  Stim.mutation_probability(mutation_probability_in_channel);
  Stim.random(random_channel);
  Stim.startGenerating(startGenerating);
  Stim.generatingDone(generatingDone);

  // Open VCD file
  sc_trace_file *tf = sc_create_vcd_trace_file("GenerationGenerator");
  tf->set_time_unit(1, SC_NS);
  sc_trace(tf, clock, "clock");
  sc_trace(tf, generation_parent1_in_channel, "generation_parent1_in_channel");
  sc_trace(tf, generation_parent2_in_channel, "generation_parent2_in_channel");
  sc_trace(tf, generation_child1_out_channel, "generation_child1_out_channel");
  sc_trace(tf, generation_child2_out_channel, "generation_child2_out_channel");
  sc_trace(tf, mutation_probability_in_channel, "mutation_probability_in_channel");
  sc_trace(tf, random_channel, "random_in_channel");
  
  reset = true;

  sc_start(2000,SC_NS);
  sc_close_vcd_trace_file(tf);
  return 0;
}
\end{lstlisting}


\begin{lstlisting}[style=customc++,caption={Stimulation for Generationgenerator},label={lst:generationGeneratiorStim}]
#pragma once
#ifdef _MSC_VER
	#include "stdafx.h"
#endif
#include "systemc.h"
#include "GenerationGenerator.h"
//#include "iostream"

SC_MODULE(Stim){
  sc_in<bool> clk;
  sc_out<sc_uint<CHROMOSOME_WIDTH> > generation_parent1;
  sc_out<sc_uint<CHROMOSOME_WIDTH> > generation_parent2;
  sc_out<sc_uint<RANDOM_WIDTH> > mutation_probability;
  sc_out<sc_uint<RANDOM_WIDTH> > random;
  sc_out<bool> startGenerating;
  sc_in<bool> generatingDone;

  void stimGen() {
    wait(10, SC_NS);
    mutation_probability->write(255);
    for(int i = 0; i < 160; i++) {
      random->write(rand()*RAND_MAX);
      wait(1, SC_NS);
    }
    generation_parent1->write(rand() * 2);
    generation_parent2->write(rand() * 2);
    startGenerating->write(true);
    wait(1, SC_MS);
  }
  SC_CTOR(Stim) {
    SC_THREAD(stimGen);
    sensitive << clk.pos();
  }
};
\end{lstlisting}

\begin{lstlisting}[style=customc++,caption={Testbench for RosenbrockSimulator},label={lst:rosenbrock_testbench}]
#ifdef _MSC_VER
	#include "stdafx.h"
#endif
#include <systemc.h>
#include "RosenbrockSimulator.h"
#include "Stim.h"

#define CLK_PERIODE   20//ns
int sc_main(int argc, char* argv[])
{
  RosenbrockSimulator RosenbrockSimulator("RosenbrockSimulator");
  Stim Stim("Stim");

  sc_clock clk("clock", sc_time(CLK_PERIODE, SC_NS)); // 50 MHz
  sc_signal<bool> reset("reset");

  sc_signal<bool> startSimulation;
  sc_signal<bool> simulationDone;
  sc_signal<sc_uint<32> > a;
  sc_signal<sc_uint<32> > b;
  sc_signal<sc_uint<32> > fitness;
  sc_signal<sc_uint<CHROMOSOME_WIDTH> > chromosome_in;

  // Wire GenerationGeneraton
  RosenbrockSimulator.clk(clk);
  RosenbrockSimulator.reset(reset);
  RosenbrockSimulator.startSimulation(startSimulation);
  RosenbrockSimulator.simulationDone(simulationDone);
  RosenbrockSimulator.a(a);
  RosenbrockSimulator.b(b);
  RosenbrockSimulator.chromosome_in(chromosome_in);
  RosenbrockSimulator.fitness(fitness);
  
  // Wire Stim
  Stim.clk(clk);
  Stim.startSimulation(startSimulation);
  Stim.simulationDone(simulationDone);
  Stim.a(a);
  Stim.b(b);
  Stim.chromosome_in(chromosome_in);
  Stim.fitness(fitness);
 
  // Open VCD file
  sc_trace_file *tf = sc_create_vcd_trace_file("GenerationGenerator");
  tf->set_time_unit(1, SC_NS);
  sc_trace(tf, clk, "clock");
  sc_trace(tf, startSimulation, "startSimulation");
  sc_trace(tf, simulationDone, "simulationDone");
  sc_trace(tf, a, "a");
  sc_trace(tf, b, "b");
  sc_trace(tf, fitness, "fitness");
  sc_trace(tf, chromosome_in, "chromosome_in");
  
  reset = true;

  sc_start(2000,SC_NS);
  sc_close_vcd_trace_file(tf);
  return 0;
}
\end{lstlisting}

\begin{lstlisting}[style=customc++,caption={Testbench for RosenbrockSimulator},label={lst:rosenbrockStim}]
#pragma once
#ifdef _MSC_VER
	#include "stdafx.h"
#endif
#include "systemc.h"
#include "RosenbrockSimulator.h"
#include "ieee754float.h"

SC_MODULE(Stim)
 {
    sc_in<bool> clk;
    sc_out<bool> startSimulation;
    sc_in<bool> simulationDone;
    sc_out<sc_uint<32> > a;//actually float
    sc_out<sc_uint<32> > b;//actually float
    sc_out<sc_uint<CHROMOSOME_WIDTH> > chromosome_in;
    sc_in<sc_uint<32> > fitness;//actually float
	
    void stimGen() {
      wait(10, SC_NS);
      a->write(floatToUint32_t(1));
      b->write(floatToUint32_t(100));
      chromosome_in->write(0x7f7fffff7f7fffff);
      wait(1, SC_NS);
      startSimulation->write(true);
      wait(10, SC_MS);
      wait(1, SC_MS);
    }
    SC_CTOR(Stim) {
      SC_THREAD(stimGen);
      sensitive << clk.pos();
    }
};
\end{lstlisting}