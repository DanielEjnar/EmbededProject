\section{Methodology}

\begin{framed}
1. Define a methodology using SysML/UML diagrams for the development of your system. Specify SysML/UML diagrams that need to be made in the different phases of the project. Make a short description of each design phases and the SysML/UML diagrams and profiles you decide to use in the methodology. Remember to use references to the papers you have use as inspiration for your work. Decide on an UML tool.
\end{framed}

The working methodology for this report will be combination of SysML \cite{sysml}(OMG 04-2006) and UML \cite{uml} (OMG 08-2006). With SysML, the non-software parts of the system can be described, e.g. the synthesized mapping onto the hardware or other register-transfer level (RTL) models. UML on the other hand provides a logical and discrete description of software classes or their interaction. As we are modeling a solution based on hardware and software co-design, both methodologies are applicable and will be described in section \ref{sec:archdesign}, architecture and design. The diagrams and models will be derived from a set of requirements put forward in section \ref{sec:req}, requirements. To quickly review the identified diagrams for this report:

\begin{itemize}
	\item Internal block diagram (IBD): To model block-specific inputs and outputs of the algorithm IP block.
	\item State machine diagram: To model the different state of the system, their transitions and guards.
	\item Activity diagram. To show the sequence of actions of the application and which activities it goes through. These are useful to get an overall idea of the use case.
	\item Class diagrams. Depicts the logical software classes, their dependencies and associations.
\end{itemize}

The analysis, design, development and validation phases follow the model-based co-design with UML/SysML described in X. This holistic process takes a starting point in the requirements, derives an architecture and design of the system, then proceeds to an implementation phase where high-level source code is designed, simulated and synthesized to a hardware description language (HDL). It concludes by conducting an entire system validation.
