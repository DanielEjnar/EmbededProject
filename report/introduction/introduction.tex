\section{Introduction}
Optimizing traditional software algorithms using hardware acceleration, like field-programmable gate array (FPGA's), is an active area of research and development due the potential decrease in latency and increase in throughput. With high-level synthesis (HLS) we refer to a process where the actual logical description of system functionality is described in a abstract modeling language, like the C-based SystemC, before being synthesized to a hardware description language like VHDL or Verilog and then executed in an FPGA on a system-on-a-chip (SoC) board. This approach makes use of a hardware and software co-design methodology to create a computational specification model in software used to generate a design that can be deployed to hardware. This brings a shorter design time, a decrease in time to market and a opportunity to evaluate many potential design options. In this report, we use HLS to describe, design and simulate a genetic optimization algorithm that finds the global minimum on the Xilinx ZyBo platform using Vivado's high-level synthesis (HLS) tool. The actual algorithm is modeled in SystemC, verified and then accelerated in an FPGA as an intellectual property (IP) core. To benchmark the algorithm, we use the two-dimensional non-convex Rosenbrock function\cite{Shang2006} given a set of user-set parameters, where the minimum resides in a parabolic-shaped valley.

Source code developed through this project is available in the zip file \textit{src.zip} with code for the two SystemC models (in folders GenerationGenerator and RosenborckSimulator) and the C++ user application (the folder userapp).