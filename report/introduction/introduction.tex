\section{Introduction}
Optimizing traditional software algorithms using hardware acceleration, like field-programmable gate array (FPGA's), is an active area of research and development due the potential decrease in latency and increase in throughput. With high-level synthesis (HLS) we refer to a process, where the actual logical description of system functionality is described in a abstract modeling language, like SystemC, before being synthesized to a hardware description language like VHDL or Verilog and then executed in a FPGA on a system-on-a-chip (SoC). This approach makes use of the hardware and software co-design methodology to create a computational specification model in software used to generate an architecture that can be deployed to hardware. This bring a shorter design time, a decrease in time to market and a opportunity to evaluate many potential design options. In this report, we use hardware and software co-design to describe, design and simulate a genetic optimization algorithm that finds global maximums or minimums on the Xiilinx ZyBo platform using Vivado's high-level synthesis (HLS) tool. The actual algorithm is modeled in SystemC, verified and then accelerated in an FPGA as an intellectual property (IP) core. To benchmark the algorithm, the non-convex Rosenbrock function is used.