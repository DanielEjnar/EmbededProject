\section{Requirements}\label{sec:req}

The following functional and non-functional requirements are stated for this report as they make up the system and architectural design. Focus is on implementing the genetic algorithm using HLS and evaluating its performance before deploying it to actual hardware. These requirements state what is expected of the system mostly, but also covers the user application.

\begin{itemize}
\item The system must be able to generate a new generation using the genetic algorithm.
\item The system must perform a simulation to evaluate the fitness of the generation using the Rosenbrock function.
\item The system should be able to generate a new generation as fast as possible.
\item The system should be able to perform a simulation of a generation as fast as possible.
\item The simulation should have user configurable parameters $a$ and $b$.
\item The user should be able to extract the generation and its fitness from the system.
\end{itemize}

The requirements are stated in the form of a use case and detailed in table \ref{tab:usecase}. It can be seen that the flow of interactions from the user's perspective is rather simple. The user simply asks for a certain Rosenbrock function using set parameters to be optimized, and gets a chromosome (bit string) that optimizes it.

\begin{table}[!htbp]
\caption{Basic use case for the system.} \label{tab:usecase}
\begin{tabular}{|l|l|}
\hline
\textbf{Goal}          & \begin{tabular}[l]{@{}l@{}}The user has gotten an optimized\\ chromosome\end{tabular}                                                                                                                                                                                                                                                                                                   \\ \hline
\textbf{Initializer}   & The user                                                                                                                                                                                                                                                                                                                                                                                \\ \hline
\textbf{Actor}         & The user                                                                                                                                                                                                                                                                                                                                                                                \\ \hline
\textbf{Prerequisits}  & \begin{tabular}[l]{@{}l@{}}The system is turned on and has\\ done its initialization.\end{tabular}                                                                                                                                                                                                                                                                                      \\ \hline
\textbf{Result}        & \begin{tabular}[l]{@{}l@{}}The user gets an optimized\\ chromosome\end{tabular}                                                                                                                                                                                                                                                                                                         \\ \hline
\textbf{Main Scenario} & \multicolumn{1}{l|}{\begin{tabular}[c]{@{}l@{}}1. The user runs the setup\\ 2. The user enters the wanted setup parameters.\\ 3. The user starts an optimization.\\ 4. The system performs the optimization.\\ 5. The system generates a new generation.\\ 6. Repeat from step 4 until stopping criterion is met.\\ 7. The user extracts the final chromosome and fitness.\end{tabular}} \\ \hline
\end{tabular}
\end{table}
 