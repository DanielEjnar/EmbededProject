\section{Requirements}\label{sec:req}

\begin{framed}
2. Write a requirement specification with functional and non-functional requirements especially with focus on performance like throughput and latency. The functional requirements can be described in terms of use cases.
\end{framed}

The following functional and non-functional are stated for this report as they make up the system and architectural design:

\begin{itemize}
\item The system must be able to generate a new generation using the genetic algorithm.
\item The system must perform a simulation to evaluate the fitness of the generation using the Rosenbrock function.
\item The system should be able to generate a new generation as fast as possible.
\item The system should be able to perform a simulation of a generation as fast as possible.
\item The simulation should have user configurable parameters $a$ and $b$.
\item The user should be able to extract the generation and its fitness from the system.
\end{itemize}

The functional non-functional requirements are stated in the form of a use case.
This use case is described in table \ref{tab:usecase}. Here it can be seen, that the flow, seen from the user perspective, is rather simple. The user basically asks for a specific Rosenbrock function to be optimized, and gets a chromosome, that optimizes it.

\begin{table}[htbp]
\caption{Basic use case for the system.} \label{tab:usecase}
\begin{tabular}{|l|l|}
\hline
\textbf{Goal}          & \begin{tabular}[l]{@{}l@{}}The user has gotten an optimized\\ chromosome\end{tabular}                                                                                                                                                                                                                                                                                                   \\ \hline
\textbf{Initializer}   & The user                                                                                                                                                                                                                                                                                                                                                                                \\ \hline
\textbf{Actor}         & The user                                                                                                                                                                                                                                                                                                                                                                                \\ \hline
\textbf{Prerequisits}  & \begin{tabular}[l]{@{}l@{}}The system is turned on and has\\ done its initialization.\end{tabular}                                                                                                                                                                                                                                                                                      \\ \hline
\textbf{Result}        & \begin{tabular}[l]{@{}l@{}}The user gets an optimized\\ chromosome\end{tabular}                                                                                                                                                                                                                                                                                                         \\ \hline
\textbf{Main Scenario} & \multicolumn{1}{l|}{\begin{tabular}[c]{@{}l@{}}1. The user runs the setup\\ 2. The user enters the wanted setup parameters.\\ 3. The user starts an optimization.\\ 4. The system performs the optimization.\\ 5. The system generates a new generation.\\ 6. Repeat from step 4 until stopping criterion is met.\\ 7. The user extracts the final chromosome and fitness.\end{tabular}} \\ \hline
\end{tabular}
\end{table}
 