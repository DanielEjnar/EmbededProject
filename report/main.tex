% arara: pdflatex: {synctex: yes, action: nonstopmode, options: "-file-line-error-style"}
% arara: bibtex
% arara: pdflatex: {synctex: yes, action: nonstopmode, options: "-file-line-error-style"}
% arara: pdflatex: {synctex: yes, action: nonstopmode, options: "-file-line-error-style"}

\documentclass{article}
\usepackage{cite}
\usepackage{url}
\usepackage[bottom]{footmisc}
\usepackage{pdfpages}
\usepackage{graphicx}
\usepackage{natbib}
\usepackage{listings}
\usepackage{color}
\usepackage{amsmath}
\usepackage{framed}
\usepackage[section]{placeins}
\lstdefinestyle{customc++}{
	belowcaptionskip=1\baselineskip,
	breaklines=true,
	frame=single,
	xleftmargin=\parindent,
	language=C++,
	captionpos=b,           % sets the caption-position to bottom
	showstringspaces=false,
	basicstyle=\footnotesize\ttfamily,
	keywordstyle=\bfseries\color{green!40!black},
	commentstyle=\itshape\color{purple!40!black},
	identifierstyle=\color{blue},
	stringstyle=\color{orange},
}
\title{Embedded Real Time Systems - Project}
\author{Daniel Ejnar Larsen (201406535) and Christian M. Lillelund(201408354)}
\date{\today}

\begin{document}
\maketitle
\begin{abstract}
High level synthesis (HLS) is an automated design process that models a computational algorithm in software and uses tools to create a hardware implementation in e.g. a field programmable gate array (FPGA). In this report, we use the event-driven SystemC language to model, stimulate and verify a genetic optimization algorithm before accelerating it in an FPGA using Xilinx's Vivado HLS tool to test its feasibility, hardware utilization and find if such algorithm would benefit from acceleration in hardware. We use HLS to design different parts of the algorithm and compare the results in terms of performance and hardware utilization. Moreover a user application for end-users are created using various design patterns and concurrent thread execution on the Zybo ZYNQ SoC Board. 
\end{abstract}

\section{Introduction}
Optimizing traditional software algorithms using hardware acceleration, like field-programmable gate array (FPGA's), is an active area of research and development due the potential decrease in latency and increase in throughput. With high-level synthesis (HLS) we refer to a process where the actual logical description of system functionality is described in a abstract modeling language, like the C-based SystemC, before being synthesized to a hardware description language like VHDL or Verilog and then executed in an FPGA on a system-on-a-chip (SoC) board. This approach makes use of the hardware and software co-design methodology to create a computational specification model in software used to generate an architecture that can be deployed to hardware. This brings a shorter design time, a decrease in time to market and a opportunity to evaluate many potential design options. In this report, we use hardware and software co-design to describe, design and simulate a genetic optimization algorithm that finds global maximums or minimums on the Xilinx ZyBo platform using Vivado's high-level synthesis (HLS) tool. The actual algorithm is modeled in SystemC, verified and then accelerated in an FPGA as an intellectual property (IP) core. To benchmark the algorithm, we use the two-dimensional non-convex Rosenbrock function\cite{Shang2006} given a set of user-set parameters, where the minimum resides in a parabolic-shaped valley.

All source code developed through this project is available in the zip file src.zip with code for the two SystemC models (in folders GenerationGenerator and RosenborckSimulator) and the C++ application(the folder userapp).

\section{Methodology}

The working methodology for this report will be a combination of SysML and UML. With SysML, the non-software parts of the system can be described, e.g. the synthesized mapping onto the hardware or other register-transfer level (RTL) models. UML on the other hand provides a logical and discrete description of software classes or their interaction. As we are modeling a solution based on hardware and software co-design, both methodologies are applicable and will be described in section \ref{sec:archdesign}, architecture and design. The diagrams and models will be derived from a set of requirements put forward in section \ref{sec:req}, requirements. To quickly review the identified diagrams for this report:

\begin{itemize}
	\item Block definition diagram (BDD): To model the overall hardware structure of the system.
	\item Internal block diagram (IBD): To model block-specific inputs and outputs of the algorithm IP block.
	\item Allocation diagram: To show possible hardware allocations of the different blocks shown in the BDD.
	\item State machine diagram: To model the different state of the system, their transitions and guards.
	\item Activity diagram. To show the sequence of actions of the application and which activities it goes through. These are useful to get an overall idea of the use case.
	\item Class diagrams. Depicts the logical software classes, their dependencies and associations.
\end{itemize}

The analysis, design, development and validation phases follow a model-based co-design with UML/SysML inspired by \cite{codesign}. This holistic process takes a starting point in the requirements, derives an architecture and design of the system, then proceeds to an implementation phase where high-level source code is designed, simulated and synthesized to a hardware description language (HDL). It concludes by conducting an entire system validation.


\section{Theory}\label{sec:theory}

This section will briefly explain the theory behind the genetic algorithm. It is defined as randomized, population based global search algorithm that tries to find global maximums or minimums of a certain function $f(x)$, in our case the Rosenbrock function: $$f(x,y) = (a-x)^2+b(y-x^2)^2$$, where $a$ and $b$ are parameters that define the function and $x$ and $y$ its coefficients. We start with an initial population $P(0)$ in which $x$ and $y$ are given as chromosomes (binary strings) of a certain size and evaluate the objective function (Rosenbrock) at points in $P(0)$ to get a fitness $F_{i}$ for each chromosome. The fitness represent a chromosome's score on the function, as a high value means we are close to either a global optimum or minimum. We save the fitness as the best-so-far chromosome and continue to create population $P_(1)$ from $P_(0)$ using the chromosomes with the best fitness so far, evaluate its fitness and so forth in an iterative manner. Given iteration $k=0$ and a initial population $P(0)$, the algorithm can be defined as\cite{Chong}:

\begin{itemize}
	\item 1. Evaluate $P(k)$. If stopping criterion is met, then stop.
	\item 2. Select $M(k) from P(k)$ as the mating pool.
	\item 3. Create new chromosomes from mating pool $M(k)$ by crossover and mutation.
	\item 4. Increase k by one and start over.
\end{itemize}

An interesting aspect is the crossover and mutation part, where we create new chromosomes (children) from binary strings (parents) with the best fitness of a population and mutate them afterwards. Figure \ref{fig:geneticalgocrossover} shows an example of a crossing, where the bit set of two parents are mixed at crossing sites to form two children. The crossing sites are chosen at random. Given two bit strings $0000$ and $1111{\color{red}4}$, the shown mutation would form two children $1100$ and $0011$. What follows is a mutation operation, where we flip bits in the children's bit string with a given probability $p_{m}$, usually a small value ($p_{m} < 0.01$), so only few chromosomes will be mutated. The idea behind these random crossings and mutations is that we seek a global maximum or minimum of our function $f(x)$ anywhere in the search space by avoiding getting stuck in local maximums or minimums, as is the case with gradient search methods. The particle swarm algorithm (PSO) shares similar attributes with genetic optimization.

\begin{figure}[h]
	\centering
	\includegraphics[width=0.7\linewidth]{theory/GeneticAlgoCrossover.png}
	\caption{An example of two parent chromosomes being crossed to create two children.}
	\label{fig:geneticalgocrossover}
\end{figure}



\section{Requirements}\label{sec:req}

\begin{framed}
2. Write a requirement specification with functional and non-functional requirements especially with focus on performance like throughput and latency. The functional requirements can be described in terms of use cases.
\end{framed}

The following functional and non-functional are stated for this report as they make up the system and architectural design:

\begin{itemize}
\item The system must be able to generate a new generation using the genetic algorithm.
\item The system must perform a simulation to evaluate the fitness of the generation using the Rosenbrock function.
\item The system should be able to generate a new generation in real-time. {\color{red} elaborate}
\item The system should be able to perform a simulation of a generation in real-time. {\color{red} elaborate}
\item The simulation should have user configurable parameters $a$ and $b$.
\item The user should be able to extract the generation and its fitness. {\color{red} really?}
\end{itemize}

The functional non-functional requirements are stated in the form of a use case.
 {\color{red} Write use case}

\section{Architecture and Design}\label{sec:archdesign}

\subsection{Architecture}
\begin{framed}
3. Use your design methodology found in 1. to describe a SysML/UML model of your system in terms of structure and behavior. Make a suggestion for alternative HW/SW architectures. Decide on which parts of the functionality should be mapped to hardware components and software processes.
\end{framed}

NOTE: waits in rosenbrock to reduce resources used


\subsection{Design}

\begin{framed}
4. Design the software and apply design patterns that are suitable for your project, and motivate the choice of the used patterns. Use the Two-Part Architecture Model if relevant for the problem. Use the abstract OS package for the ZYBO board
\end{framed}

\section{Modelling}

\begin{framed}
5. Select the SysML/UML model for your preferred architecture suggestions in 3. Choose a part of the functionality including both hardware and software components to create a model and a test- bench in HLS using SystemC or C-code. Simulate and validate your design model.

Argue for your choice of modeling language and abstraction level of modeling. Use the reports from the HLS tool to evaluate performance of the design. Assess whether the design is able to fulfill your requirements and constraints.
\end{framed}

%\section{SystemC Modelling}

%For making the model of the software SystemC is used. SystemC is a C++ framework? for programming a model of logical hardware such as an FPGA. 



\section{User application in C++}\label{sec:userapp}

This section will briefly explain how we implemented the user application in C++. The program is used by the user via a terminal, and the user creates actions using standard input. Two concurrent OSAPI threads are created in FreeRTOS, one to handle user input and one to create the random numbers continually. Figure \ref{lst:mainapp} is a short code snippet of the main application.

\begin{lstlisting}[style=customc++,caption={Main application, where two threads are created and the scheduler started.},label={lst:mainapp}]
int main()
{
	//Create threads
	UserThread mUserThread(Thread::PRIORITY_ABOVE_NORMAL, "UserThread");
	RandomThread mRandomhread(Thread::PRIORITY_NORMAL, "RandomThread");

	/* Start FreeRTOS, the tasks running. */
	vTaskStartScheduler();
	for( ;; );
	return 0;
}
\end{lstlisting}

The $UserThread$ is the main point of interest. It has a virtual function $run()$ executed by FreeRTOS and shown in listing \ref{lst:mainrun} containing the execution loop. We chose to omit the $RandomThread$ from this review, however the code can be found in the attached code (userapp/RandomThread.h). We create a smart pointer to the \textbf{Context} class, as the client interacts solely with the states and data through this. Using smart pointers improve safety of the application in case of exceptions and they will be automatically deleted when exiting program scope. 

Given a certain input, we create a pointer of class \textbf{Action} with a sting of what is to be done. This is evaluated by the current state of the application, as seen in the example in listing \ref{lst:mainrun}.

\begin{lstlisting}[style=customc++,caption={Main application, where two threads are created with a scheduler.},label={lst:mainrun}]
{
	std::unique_ptr<Context> context = std::make_unique<Context>();
	bool running = true;
	char* str;
	while(running) {
		std::cout << "Welcome to the menu" << std::endl;
		std::cout << "Press 1 for SETUP, 2 for RUN, 3 for ABORT" << std::endl;
		std::cin >> str;
		int num = atoi(str);
		switch(num) {
			case 1:
				context->HandleInput(std::make_unique<Action>("ENTER_SETUP"));
				context->HandleInput(std::make_unique<Action>("SETUP_DONE"));
				break;
			case 2:
				context->HandleInput(std::make_unique<Action>("OPTIMIZE"));
				break;
			case 3:
				context->HandleInput(std::make_unique<Action>("ABORT"));
		}
	}
}
\end{lstlisting}

The \textbf{Context} contains various functionality, but of most interest is the $HandleInput()$ function. It takes an action from user space, delegates it to the current state and expects a pointer returned to an eventual new state. The current state is then updated with the returned state, in between calls to $Exit()$ and $Enter()$. Listing \ref{lst:context} gives an example in C++ code.

\begin{lstlisting}[style=customc++,caption={The Context class holds the current state among other variables, such as best chromosome so far and parameters. Here is showed the function HandleInput() called from user side.},label={lst:context}]
void Context::HandleInput(std::unique_ptr<Action> action) {
	std::unique_ptr<State> newState = _currentState->HandleAction(*this,
		std::move(action));
	if (newState != NULL) {
		_currentState->Exit(*this);
		_currentState.reset();
		_currentState = std::move(newState);
		_currentState->Enter(*this);
	}
}
\end{lstlisting}

To complete our example, we review the $HandleAction()$ function in \textbf{Idle} class listing \ref{lst:stateidle}. It takes the Context and some action as parameters, checks the content of the action and acts accordingly by returning a pointer to the next state the application will transition to, in this case \textbf{Setup}.

\begin{lstlisting}[style=customc++,caption={The HandleAction() function in state Idle.},label={lst:stateidle}]
std::unique_ptr<State> Idle::HandleAction(Context& context, std::unique_ptr<Action> action)
{
	if((*action).GetAction() == "ENTER_SETUP") {
		std::cout << "EnterSetup() called" << std::endl;
		action.reset();
		return std::make_unique<Setup>();
	}
	...
	return NULL;
}
\end{lstlisting}


\section{Results}\label{sec:results}
\begin{framed}
6. Implement and test a part of your system using the ZYBO platform including at least one IP core written and verified with the HLS tool.
\end{framed}

This section shows and elaborates on the results in performance and utilization of all the designs we purposed in section \ref{sec:archdesign}. This includes a high-level synthesis report of the two IP cores (GenerationGenerator and Simulator) and the user application, that was implemented in C++ and deployed to the ZYBO board with the use of OSAPI threads and FreeRTOS. Overall, our results show that simple fixed-point arithmetic can be accelerated in an hardware FPGA at relative low cost, however more complicated floating point computations are better done in software, as the number of required DSP's and lookup tables rapidly increase in such cases.

\subsection{HLS of the GenerationGenerator core}

The IP core was synthesized in C using a VivadoSimulator, a period of 20 and Verilog as the RTL language. Target device is a Zybo model xc7z010clg400-1. We used a SystemC stimuli file to invoke the IP core, which can be found in Appendix A. Figure \ref{fig:ggperformanceestimates} shows the performance estimates of the applied design. We seen that the estimated clock is well within our target of 20 with an uncertainty of 2.50 and the latency lie in range of 270 to 271.

\begin{figure}[htbp]
	\centering
	\fbox{\includegraphics[width=1.0\linewidth]{../diagrams/GGperformanceEstimates.png}}
	\caption{X}
	\label{fig:ggperformanceestimates}
\end{figure}

Figure \ref{fig:ggutilizationestimates} shows utilization of hardware estimates. We see that the demand of resources are well within range of what is available on the Zybo board. 

\begin{figure}[htbp]
	\centering
	\fbox{\includegraphics[width=1.0\linewidth]{../diagrams/GGutilizationEstimates.png}}
	\caption{X}
	\label{fig:ggutilizationestimates}
\end{figure}

Figure \ref{fig:gginterface} shows the generated interface signals. These echo the interfaces we defined in the definition of source file. Direction indicate if they are to be read or written to. The bit length stems from the width of a chromosome.

\begin{figure}[htbp]
	\centering
	\fbox{\includegraphics[width=1.0\linewidth]{../diagrams/GGinterface}}
	\caption{}
	\label{fig:gginterface}
\end{figure}

Figure \ref{fig:generationgeneratortrace} shows the trace output after running a simulation with a stimuli. We seen that the random input data changes continually, and when the generating core is started, it uses the two parents and a mutation probability to create new children chromosomes and output these.

\begin{figure}[htbp]
	\centering
	\fbox{\includegraphics[width=1\linewidth]{../diagrams/GenerationGeneratorSim.png}}
	\caption{A trace diagram that shows all bidirectional signals of the module during a example stimulation.}
	\label{fig:generationgeneratortrace}
\end{figure}

\subsection{HLS of the Simulator core}

The IP was synthesized with the same settings as the GenerationGenertor in the Vivado HLS tool. We used a SystemC stimuli file to make calls to the IP core, which can be found in Appendix A. Figure \ref{fig:simperformanceestimates} shows performance estimates, with the estimated clock being within the range of the target clock and the latency being from  0 to 96. 

\begin{figure}[htbp]
	\centering
	\fbox{\includegraphics[width=1.0\linewidth]{../diagrams/SimperformanceEstimates.png}}
	\caption{X}
	\label{fig:simperformanceestimates}
\end{figure}

In figure \ref{fig:ggutilizationestimates} we see the utilization estimates of running the simulation in hardware. The total resource demand exceed those of the platform, the Zybo board, by quite a significant margin. The algorithm need 592 DSP48E's, but only 80 are available. Likewise it needs 39988 flip-flops and 22918 lookup tables, but fewer are available. This means that the current implementation will not be runnable on hardware, thus we decided not to make the simulator in hardware but in software instead and make it part of the user application.

\begin{figure}[htbp]
	\centering
	\fbox{\includegraphics[width=1.0\linewidth]{../diagrams/SimutilizationEstimates.png}}
	\caption{X}
	\label{fig:simutilizationestimates}
\end{figure}

Figure \ref{fig:siminterface} shows the generated interface signals with the RTL ports and their respective direction and bit length. 

\begin{figure}[htbp]
	\centering
	\fbox{\includegraphics[width=1.0\linewidth]{../diagrams/Siminterface}}
	\caption{X}
	\label{fig:siminterface}
\end{figure}}


\section{Conclusion}
In conclusion this project has been of a some success. We have designed and implemented two different IP cores using SystemC - one which generates a new generation of chromosomes and one which simulates the Rosenbrock function. Both of these were verified using a C simulation, but due to tool problems we ere unable to make a C/RTL co-simulation. Even though it was unable to co-simulate both of the modules were able to synthesize and gave the expected interfaces. 

As described in section \ref{sec:results} only one of the IP cores, GenerationGenerator, were able to fit on the FPGA. That our IP core RosenbrockSimulator did not fit on the FPGA shows, that it is troublesome to use floats even though it was only internally in the module. We could have used a different encoding of the floating point numbers, which could use fixed point arithmetics, but we did not have the time to implement it.
This design with the ZYNQ processor and GenerationGenerator were synthesized to the FPGA and we were able to run software on it.

The software used a concurrent state pattern with a command pattern to change the states. This was implemented with great success and we were able to run the software on the ZyBo board after having some troubles with getting STDIO to go to our terminal.

The software changed through the states as we expected and were able to recognize and react correctly to all of the commands.





\bibliographystyle{plain}
\bibliography{bibliography}

\appendix

%\section{Code} \label{app:code}

In the following the code for the testbenches and the stimulus are shown.

\begin{lstlisting}[style=customc++,caption={Testbench for GenerationGenerator},label={lst:generatorTestbench}]
#ifdef _MSC_VER
	#include "stdafx.h"
#endif
#include <systemc.h>
#include "GenerationGenerator.h"
#include "Stim.h"

#define CLK_PERIODE   20//ns
int sc_main(int argc, char* argv[]) {
  GenerationGenerator GenerationGenerator("GenerationGenerator");
  Stim Stim("Stim");
  
  sc_clock clock("clock", sc_time(CLK_PERIODE, SC_NS)); // 50 MHz
  sc_signal<bool> reset("reset");

  sc_signal<sc_uint<CHROMOSOME_WIDTH> > generation_parent1_in_channel;
  sc_signal<sc_uint<CHROMOSOME_WIDTH> > generation_parent2_in_channel;
  sc_signal<sc_uint<CHROMOSOME_WIDTH> > generation_child1_out_channel;
  sc_signal<sc_uint<CHROMOSOME_WIDTH> > generation_child2_out_channel;
  sc_signal<sc_uint<RANDOM_WIDTH> > mutation_probability_in_channel("mutation_probability_in_channel");
  sc_signal<sc_uint<RANDOM_WIDTH> > random_channel("random_channel");
  sc_signal<bool> generatingDone;
  sc_signal<bool> startGenerating;

  // Wire GenerationGeneraton
  GenerationGenerator.generation_parent1(generation_parent1_in_channel);
  GenerationGenerator.generation_parent2(generation_parent2_in_channel);
  GenerationGenerator.generation_child1(generation_child1_out_channel);
  GenerationGenerator.generation_child2(generation_child2_out_channel);
  GenerationGenerator.clk(clock);
  GenerationGenerator.mutation_probability(mutation_probability_in_channel);
  GenerationGenerator.random(random_channel);
  GenerationGenerator.reset(reset);
  GenerationGenerator.startGenerating(startGenerating);
  GenerationGenerator.generatingDone(generatingDone);

  // Wire Stim
  Stim.generation_parent1(generation_parent1_in_channel);
  Stim.generation_parent2(generation_parent2_in_channel);
  Stim.clk(clock);
  Stim.mutation_probability(mutation_probability_in_channel);
  Stim.random(random_channel);
  Stim.startGenerating(startGenerating);
  Stim.generatingDone(generatingDone);

  // Open VCD file
  sc_trace_file *tf = sc_create_vcd_trace_file("GenerationGenerator");
  tf->set_time_unit(1, SC_NS);
  sc_trace(tf, clock, "clock");
  sc_trace(tf, generation_parent1_in_channel, "generation_parent1_in_channel");
  sc_trace(tf, generation_parent2_in_channel, "generation_parent2_in_channel");
  sc_trace(tf, generation_child1_out_channel, "generation_child1_out_channel");
  sc_trace(tf, generation_child2_out_channel, "generation_child2_out_channel");
  sc_trace(tf, mutation_probability_in_channel, "mutation_probability_in_channel");
  sc_trace(tf, random_channel, "random_in_channel");
  
  reset = true;

  sc_start(2000,SC_NS);
  sc_close_vcd_trace_file(tf);
  return 0;
}
\end{lstlisting}


\begin{lstlisting}[style=customc++,caption={Stimulation for Generationgenerator},label={lst:generationGeneratiorStim}]
#pragma once
#ifdef _MSC_VER
	#include "stdafx.h"
#endif
#include "systemc.h"
#include "GenerationGenerator.h"
//#include "iostream"

SC_MODULE(Stim){
  sc_in<bool> clk;
  sc_out<sc_uint<CHROMOSOME_WIDTH> > generation_parent1;
  sc_out<sc_uint<CHROMOSOME_WIDTH> > generation_parent2;
  sc_out<sc_uint<RANDOM_WIDTH> > mutation_probability;
  sc_out<sc_uint<RANDOM_WIDTH> > random;
  sc_out<bool> startGenerating;
  sc_in<bool> generatingDone;

  void stimGen() {
    wait(10, SC_NS);
    mutation_probability->write(255);
    for(int i = 0; i < 160; i++) {
      random->write(rand()*RAND_MAX);
      wait(1, SC_NS);
    }
    generation_parent1->write(0xDEADDEADBEEFBEEF);
    generation_parent2->write(0x1234567812345678);
    startGenerating->write(true);
    wait(1, SC_MS);
  }
  SC_CTOR(Stim) {
    SC_THREAD(stimGen);
    sensitive << clk.pos();
  }
};
\end{lstlisting}

\begin{lstlisting}[style=customc++,caption={Testbench for RosenbrockSimulator},label={lst:rosenbrock_testbench}]
#ifdef _MSC_VER
	#include "stdafx.h"
#endif
#include <systemc.h>
#include "RosenbrockSimulator.h"
#include "Stim.h"

#define CLK_PERIODE   20//ns
int sc_main(int argc, char* argv[])
{
  RosenbrockSimulator RosenbrockSimulator("RosenbrockSimulator");
  Stim Stim("Stim");

  sc_clock clk("clock", sc_time(CLK_PERIODE, SC_NS)); // 50 MHz
  sc_signal<bool> reset("reset");

  sc_signal<bool> startSimulation;
  sc_signal<bool> simulationDone;
  sc_signal<sc_uint<32> > a;
  sc_signal<sc_uint<32> > b;
  sc_signal<sc_uint<32> > fitness;
  sc_signal<sc_uint<CHROMOSOME_WIDTH> > chromosome_in;

  // Wire GenerationGeneraton
  RosenbrockSimulator.clk(clk);
  RosenbrockSimulator.reset(reset);
  RosenbrockSimulator.startSimulation(startSimulation);
  RosenbrockSimulator.simulationDone(simulationDone);
  RosenbrockSimulator.a(a);
  RosenbrockSimulator.b(b);
  RosenbrockSimulator.chromosome_in(chromosome_in);
  RosenbrockSimulator.fitness(fitness);
  
  // Wire Stim
  Stim.clk(clk);
  Stim.startSimulation(startSimulation);
  Stim.simulationDone(simulationDone);
  Stim.a(a);
  Stim.b(b);
  Stim.chromosome_in(chromosome_in);
  Stim.fitness(fitness);
 
  // Open VCD file
  sc_trace_file *tf = sc_create_vcd_trace_file("GenerationGenerator");
  tf->set_time_unit(1, SC_NS);
  sc_trace(tf, clk, "clock");
  sc_trace(tf, startSimulation, "startSimulation");
  sc_trace(tf, simulationDone, "simulationDone");
  sc_trace(tf, a, "a");
  sc_trace(tf, b, "b");
  sc_trace(tf, fitness, "fitness");
  sc_trace(tf, chromosome_in, "chromosome_in");
  
  reset = true;

  sc_start(2000,SC_NS);
  sc_close_vcd_trace_file(tf);
  return 0;
}
\end{lstlisting}

\begin{lstlisting}[style=customc++,caption={Testbench for RosenbrockSimulator},label={lst:rosenbrockStim}]
#pragma once
#ifdef _MSC_VER
	#include "stdafx.h"
#endif
#include "systemc.h"
#include "RosenbrockSimulator.h"
#include "ieee754float.h"

SC_MODULE(Stim)
 {
    sc_in<bool> clk;
    sc_out<bool> startSimulation;
    sc_in<bool> simulationDone;
    sc_out<sc_uint<32> > a;//actually float
    sc_out<sc_uint<32> > b;//actually float
    sc_out<sc_uint<CHROMOSOME_WIDTH> > chromosome_in;
    sc_in<sc_uint<32> > fitness;//actually float
	
    void stimGen() {
      wait(10, SC_NS);
      a->write(floatToUint32_t(1));
      b->write(floatToUint32_t(100));
      chromosome_in->write(0x7f7fffff7f7fffff);
      wait(1, SC_NS);
      startSimulation->write(true);
      wait(10, SC_MS);
      wait(1, SC_MS);
    }
    SC_CTOR(Stim) {
      SC_THREAD(stimGen);
      sensitive << clk.pos();
    }
};
\end{lstlisting}

\end{document}
