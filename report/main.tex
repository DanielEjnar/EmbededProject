\documentclass{article}
\usepackage{cite}
\usepackage{url}
\usepackage[bottom]{footmisc}
\usepackage{pdfpages}
\usepackage{graphicx}
\usepackage{natbib}
\usepackage{listings}
\usepackage{color}
\lstdefinestyle{customc++}{
	belowcaptionskip=1\baselineskip,
	breaklines=true,
	numbers=left,
	frame=single,
	xleftmargin=\parindent,
	language=C++,
	captionpos=b,           % sets the caption-position to bottom
	showstringspaces=false,
	basicstyle=\footnotesize\ttfamily,
	keywordstyle=\bfseries\color{green!40!black},
	commentstyle=\itshape\color{purple!40!black},
	identifierstyle=\color{blue},
	stringstyle=\color{orange},
}
\title{Embedded Real Time Systems - Project}
\author{Daniel Ejnar and Christian M. Lillelund}
\date{\today}
\begin{document}
\maketitle
\begin{abstract}
	In this report, a genetic optimization algorithm is modeled in SystemC, then verified and accelerated in an FPGA using Vivado's HLS tool to find if this kind of global search algorithm benefits from hardware acceleration. The results of comparing the algorithm evaluated on a standard general-purpose processor with the hardware accelerated version in terms of performance, energy usage and space are presented.
\end{abstract}

\section{Introduction}
Optimizing traditional software algorithms using hardware acceleration, like field-programmable gate array (FPGA's), is an active area of research and development due the potential decrease in latency and increase in throughput. With high-level synthesis (HLS) we refer to a process, where the actual logical description of system functionality is described in a abstract modeling language, like SystemC, before being synthesized to a hardware description language like VHDL or Verilog and then executed in a FPGA on a system-on-a-chip (SoC). This approach makes use of the hardware and software co-design methodology to create a computational specification model in software used to generate an architecture that can be deployed to hardware. This bring a shorter design time, a decrease in time to market and a opportunity to evaluate many potential design options. In this report, we use hardware and software co-design to describe, design and simulate a genetic optimization algorithm that finds global maximums or minimums on the Xiilinx ZyBo platform using Vivado's high-level synthesis (HLS) tool. The actual algorithm is modeled in SystemC, verified and then accelerated in an FPGA as an intellectual property (IP) core. To benchmark the algorithm, the non-convex Rosenbrock function is used.

\section{Methodology}

The working methodology for this report will be combination of SysML (OMG 04-2006) and UML (OMG 08-2006). With SysML, the non-software parts of the system can be described, e.g. the synthesized mapping onto the hardware or other register-transfer level (RTL) models. UML on the other hand provides a logical and discrete description of software classes or their interaction. As we are modeling a solution based on hardware and software co-design, both methodologies are applicable and will be described in section X, architecture and design. The diagrams and models will be derived from a set of requirements put forward in section X, requirements. To quickly review the identified diagrams for this report:

\begin{itemize}
	\item Internal block diagram (IBD): To model block-specific inputs and outputs of the algorithm IP block.
	\item Activity diagram. To show the sequence of actions of the application and which activities it goes through. These are useful to get an overall idea of the use case.
	\item Class diagrams. Depicts the logical software classes, their dependencies and associations.
\end{itemize}

The analysis, design, development and validation phases follow the model-based codesign with UML/SysML described in X. This holistic process takes a starting point in the requirements, derives an architecture and design of the system, then proceeds to an implementation phase where high-level source code is designed, simulated and synthesized to a hardware description language (HDL). It concludes by conducting an entire system validation.

\section{Requirements}

The following functional and non-functional are stated for this report as they make up the system and architectural design:

\section{Architecture and design}




\section{Theory and concepts}
\section{Modeling in SystemC}
\section{Simulation results}
\section{Conclusion}
\bibliographystyle{pl/ain}
\bibliography{bibtex}{}
\end{document}
