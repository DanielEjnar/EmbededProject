\section{Conclusion}
In conclusion this project has been of a some success. We have managed to design and implement two different IP cores using SystemC - one which generates a new generation of chromosomes and one which simulates the Rosenbrock function. Both of these were verified using a C simulation, but due to tool problems we ere unable to make a C/RTL co-simulation, however both modules were able to synthesize and gave the expected interfaces. Unfortunately the GenerationGenerator IP core were unable to give an output, and due to the tool issues we were unable to co-simulate, and thereby investigate where why it did not give an output. Next step would be to investigate these issues by isolating logic and simplify the cores.

As described in section \ref{sec:results} only one of the IP cores, GenerationGenerator, were able to fit on the FPGA. The IP core RosenbrockSimulator did not fit on the FPGA because of the use of floats that require additional hardware overhead tha fixed-point arithmetic, even though they were only used internally in the module. We could have used a different encoding of the floating point numbers, which could use fixed point arithmetics, but we did not have the time to implement it. This design with the ZYNQ processor and GenerationGenerator were synthesized to the FPGA and we were able to run software on it.

The software used a concurrent state pattern with a command pattern to change the states. This was implemented with great success and we were able to run the software on the ZyBo board after having some troubles with getting STDIO to go to our terminal, which was a issue with the Vivado SDK tool. As expected the application cycled through the states and it was able to recognize and react correctly to all commands.
