\section{Theory}\label{sec:theory}

This section will briefly explain the theory behind the genetic algorithm. It is defined as a randomized, population based global search algorithm that tries to find a global maximum or minimum of a certain function $f(x)$, in our case the Rosenbrock function: $$f(x,y) = (a-x)^2+b(y-x^2)^2$$ where $a$ and $b$ are parameters that define the function and $x$ and $y$ its coefficients. We start with an initial population $P_{(0)}$ in which $x$ and $y$ are given as chromosomes (binary strings) of a certain size and evaluate the objective function (the Rosenbrock function\cite{Shang2006}) at points in $P_{(0)}$ to get a fitness $F_{i}$ for each chromosome. The fitness represent a chromosome's score on the function, as a high value means we are close to either a global optimum or minimum. We save the fitness as the best-so-far chromosome and continue to create population $P_{(1)}$ from $P_{(0)}$ using the chromosomes with the best fitness so far, evaluate its fitness and so forth in an iterative manner. Given iteration $k=0$ and a initial population $P_{(0)}$, the algorithm can be defined as\cite{Chong2013}:

\begin{enumerate}
	\item Evaluate $P_{(k)}$. If stopping criterion is met, then stop.
	\item Select $M_{(k)}$ from $P_{(k)}$ as the mating pool.
	\item Create new chromosomes from mating pool $M_{(k)}$ by crossover and mutation.
	\item Increase $k$ by one and start over.
\end{enumerate}

An interesting aspect is the crossover and mutation part, where we create new chromosomes (children) from binary strings (parents) with the best fitness of a population and mutate them afterwards. Figure \ref{fig:geneticalgocrossover} shows an example of a crossing, where the bit set of two parents are mixed at crossing sites to form two children. The crossing sites are chosen at random. Given two bit strings, e.g. $0000$ and $1111$, the mutation shown in figure \ref{fig:geneticalgocrossover} would form two children $1001$ and $0110$. What follows is a mutation operation, where we flip bits in the children's bit string with a given probability $p_{m}$, usually a small value ($p_{m} < 0.01$), so only few chromosomes will be mutated. The idea behind these random crossings and mutations is that we seek a global maximum or minimum of our function $f(x)$ anywhere in the search space by avoiding getting stuck in local maxima or minima, as is a potential outcome with gradient search methods. The particle swarm algorithm (PSO) shares similar attributes to genetic optimization.

\begin{figure}[h]
	\centering
	\fbox{\includegraphics[width=0.7\linewidth]{theory/GeneticAlgoCrossover.png}}
	\caption{An example of two parent chromosomes being crossed to create two children.}
	\label{fig:geneticalgocrossover}
\end{figure}


